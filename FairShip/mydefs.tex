%--------------------
% Macros for Journals
%--------------------
%\def\Journal#1#2#3#4{{#1} {\bf #2}, #3 (#4)}
\def\Journal#1#2#3#4{{#1} {#2} (#3) #4}
\def\ARN{\em Ann. Rev. Nucl. Part. Sci.}
\def\CPC{\em Comp. Phys. Comm.}
\def\EPC{{\em Eur. Phys. J.} C}
\def\IJM{{\em Int. J. Mod. Phys.} A}
\def\JPC{{\em Eur. Phys. J.} C}
\def\JPG{{\em J. Phys.} G}
\def\NCA{\em Nuovo Cimento}
\def\NIM{\em Nucl. Instrum. Methods}
\def\NIM{{\em Nucl. Instr. and Meth.} A}
\def\NPB{{\em Nucl. Phys.} B}
\def\PLB{{\em Phys. Lett.}  B}
\def\PRE{\em Phys. Rep.}
\def\PRD{{\em Phys. Rev.} D}
\def\PRL{\em Phys. Rev. Lett.}
\def\SJN{\em Sov. J. Nucl. Phys.}
\def\SPJ{\em Sov. Phys. JETP}
\def\ZPC{{\em Z. Phys.} C}
\def\ZPX{\em Z. Phys.}

\def\etal{\textit{et al.}}

%%%%%%%%%%%%%%%%%%%%%%%%%%%%%%%%%%%%%%%%%%%%%%%%%%
% Units
%%%%%%%%%%%%%%%%%%%%%%%%%%%%%%%%%%%%%%%%%%%%%%%%%%
\newcommand{\unit}[1]{\ensuremath{\rm\,#1}\xspace}          % {kg}

%% Energy and momentum
\newcommand{\tev}{\ensuremath{\mathrm{\,Te\kern -0.1em V}}\xspace}
\newcommand{\gev}{\ensuremath{\mathrm{\,Ge\kern -0.1em V}}\xspace}
\newcommand{\GeV}{\ensuremath{\mathrm{\,Ge\kern -0.1em V}}\xspace}
\newcommand{\mev}{\ensuremath{\mathrm{\,Me\kern -0.1em V}}\xspace}
\newcommand{\kev}{\ensuremath{\mathrm{\,ke\kern -0.1em V}}\xspace}
\newcommand{\ev}{\ensuremath{\mathrm{\,e\kern -0.1em V}}\xspace}
\newcommand{\gevc}{\ensuremath{{\mathrm{\,Ge\kern -0.1em V\!/}c}}\xspace}
\newcommand{\mevc}{\ensuremath{{\mathrm{\,Me\kern -0.1em V\!/}c}}\xspace}
\newcommand{\gevcc}{\ensuremath{{\mathrm{\,Ge\kern -0.1em V\!/}c^2}}\xspace}
\newcommand{\gevgevcccc}{\ensuremath{{\mathrm{\,Ge\kern -0.1em V^2\!/}c^4}}\xspace}
\newcommand{\mevcc}{\ensuremath{{\mathrm{\,Me\kern -0.1em V\!/}c^2}}\xspace}
\newcommand{\phim}     { \ensuremath{\mathrm{ e^{i\varphi_M} } }\xspace}
\newcommand{\phims}    { \ensuremath{\mathrm{ e^{-i\varphi_M} } }\xspace}
\newcommand{\ii}{\ensuremath{i}}
\newcommand{\mez}{\ensuremath{M_{12}}}
\newcommand{\gez}{\ensuremath{\Gamma_{12}}}
\newcommand{\vghost}   { \rule[-1ex]{0mm}{3.2ex}}
\newcommand{\vghostu}   { \rule[0ex]{0mm}{2.5ex}}
\newcommand{\vghostd}   { \rule[-1ex]{0mm}{0ex}}
\newcommand{\vghostm}  { \rule[0ex]{0mm}{2.5ex}}
\newcommand{\abag}{$\alpha$, $\beta$ and $\gamma$}
\newcommand{\abg}{$\alpha$, $\beta$, $\gamma$}
\newcommand{\dlam}     { \ensuremath{ \Delta \lambda} \xspace}
\newcommand{\Exp}      { \ensuremath{ \mathrm{e}       }}
\newcommand{\epsi}     { \ensuremath{\mathrm{\varepsilon_T}}}
\newcommand{\phisw}    { \ensuremath{\mathrm{ \varphi_{SW} } }\xspace}
\newcommand{\atan}     { \ensuremath{\mathrm{atan}}}
\newcommand{\egq}      { \ensuremath{\mathrm{ e^{-\overline{\Gamma} \tau}}}}
\newcommand{\nd}[1]{/\hspace{-0.6em} #1}
\newcommand{\PKz}{\ensuremath{\mathrm{K^0}}}
\newcommand{\PaKz}{\ensuremath{\mathrm{\overline{K}\mbox{}^0}}}
\newcommand{\nn}{\nonumber}

%------------------------------------------------------------------------------
%       -> (anti-)something with correct hyphenation
%------------------------------------------------------------------------------
\newcommand{\Anti}[1]{\hbox{(anti-)}\discretionary{}{}{}#1}
\newcommand{\anti}[1]{(anti\discretionary{-)}{}{)}#1}
\newcommand{\CL}[1]{$#1\%$~C.L\xdotspace}
%------------------------------------------------------------------------------
%       -> vertical rule of width zero and variable height & depth
%------------------------------------------------------------------------------
\newcommand{\stru}[2]{%
   \relax\ifmmode\hbox{\vrule height#1 depth#2 width0pt}%
   \else\vrule height#1 depth#2 width0pt\fi}
%------------------------------------------------------------------------------
%       -> underline with resonable distance text - line
%------------------------------------------------------------------------------
\newcommand{\uline}[1]{$\underline{\hbox{#1\stru{0.pt}{.485ex}}}$}
%------------------------------------------------------------------------------
%       -> roman numbers (uppercase and lowercase)
%------------------------------------------------------------------------------
\newcommand{\Ronum}[1]{\uppercase\expandafter{\romannumeral#1}}
\newcommand{\ronum}[1]{\expandafter{\romannumeral#1}}
%------------------------------------------------------------------------------
%       -> shorthands for equation, figure, table, section references
%          (require standard use of labels: eq-... for equations, cha-...
%           for chapters, sec-... for (sub)sections, tab-... for tables
%           and fig-... for figures)
%------------------------------------------------------------------------------
\newcommand{\eq}[1]{(\ref{eq-#1})}
\newcommand{\eqsand}[2]{(\ref{eq-#1}) and~(\ref{eq-#2})}
\newcommand{\eqsto}[2]{(\ref{eq-#1})--(\ref{eq-#2})}
\newcommand{\eqstwo}[2]{(\ref{eq-#1},\ref{eq-#2})}
\newcommand{\eqsthr}[3]{(\ref{eq-#1},\ref{eq-#2},\ref{eq-#3})}
\newcommand{\eqsfou}[4]{(\ref{eq-#1},\ref{eq-#2},\ref{eq-#3},\ref{eq-#4})}
\newcommand{\Eq}[1]{Equation~(\ref{eq-#1})}
\newcommand{\Eqsto}[2]{Equations~(\ref{eq-#1})--(\ref{eq-#2})}
\newcommand{\fig}[1]{Fig.~\ref{fig-#1}}
\newcommand{\Fig}[1]{Figure~\ref{fig-#1}}
\newcommand{\figand}[2]{Figs.~\ref{fig-#1} and~\ref{fig-#2}}
\newcommand{\Figand}[2]{Figures~\ref{fig-#1} and~\ref{fig-#2}}
\newcommand{\tab}[1]{Table~\ref{tab-#1}}
\newcommand{\Tab}[1]{Table~\ref{tab-#1}}
\newcommand{\taband}[2]{Tables~\ref{tab-#1} and~\ref{tab-#2}}
\newcommand{\cha}[1]{Chap.~\ref{cha-#1}}
\newcommand{\Cha}[1]{Chapter~\ref{cha-#1}}
\newcommand{\sect}[1]{Sect.~\ref{sec-#1}}
\newcommand{\Sect}[1]{Section~\ref{sec-#1}}
\newcommand{\sectand}[2]{Sects.~\ref{sec-#1} and~\ref{sec-#2}}
\newcommand{\sectto}[2]{Sects.~\ref{sec-#1} to~\ref{sec-#2}}
\newcommand{\Sectand}[2]{Sections~\ref{sec-#1} and~\ref{sec-#2}}
\newcommand{\Sectto}[2]{Sections~\ref{sec-#1} to~\ref{sec-#2}}
%------------------------------------------------------------------------------
%       -> log file message
%------------------------------------------------------------------------------
\newcommand{\LogMess}[1]{\typeout{^^J==> #1^^J}}
%------------------------------------------------------------------------------
%       Some special symbols and settings for math mode:
%       ------------------------------------------------
%       -> redefine mathbf
%------------------------------------------------------------------------------
\DeclareMathAlphabet{\mathbf}{OT1}{cmr}{bx}{sl}

%------------------------------------------------------------------------------
%       -> various other units
%------------------------------------------------------------------------------
\newcommand{\mb}{\,\text{mb}}
\newcommand{\mub}{\,\upmu\text{b}}
\newcommand{\nb}{\,\text{nb}}
\newcommand{\pb}{\,\text{pb}}
\newcommand{\fb}{\,\text{fb}}
\newcommand{\mubi}{\,\upmu\text{b}^{-1}}
\newcommand{\mbi}{\,\text{mb}^{-1}}
\newcommand{\nbi}{\,\text{nb}^{-1}}
\newcommand{\pbi}{\,\text{pb}^{-1}}
\newcommand{\fbi}{\,\text{fb}^{-1}}
\newcommand{\met}{\,\text{m}}
\newcommand{\nm}{\,\text{nm}}
\newcommand{\mum}{\,\upmu\text{m}}
\newcommand{\mm}{\,\text{mm}}
\newcommand{\cm}{\,\text{cm}}
\newcommand{\km}{\,\text{km}}
\newcommand{\Amp}{\,\text{A}}
\newcommand{\nA}{\,\text{nA}}
\newcommand{\muA}{\,\upmu\text{A}}
\newcommand{\mA}{\,\text{mA}}
\newcommand{\kA}{\,\text{kA}}
\newcommand{\Hz}{\,\text{Hz}}
\newcommand{\kHz}{\,\text{kHz}}
\newcommand{\MHz}{\,\text{MHz}}
\newcommand{\GHz}{\,\text{GHz}}
\newcommand{\ppm}{\,\text{ppm}}
\newcommand{\ppb}{\,\text{ppb}}
\newcommand{\scd}{\,\text{s}}
\newcommand{\ps}{\,\text{ps}}
\newcommand{\ns}{\,\text{ns}}
\newcommand{\mus}{\,\upmu\text{s}}
\newcommand{\ms}{\,\text{ns}}
\newcommand{\rad}{\,\text{rad}}
\newcommand{\murad}{\,\upmu\text{rad}}
\newcommand{\mrad}{\,\text{mrad}}
\newcommand{\gra}{\,\text{g}}
\newcommand{\pg}{\,\text{pg}}
\newcommand{\nang}{\,\text{ng}}
\newcommand{\mug}{\,\upmu\text{g}}
\newcommand{\mg}{\,\text{mg}}
\newcommand{\kg}{\,\text{kg}}
\newcommand{\Tesla}{\,\text{T}}
\newcommand{\Kelvin}{\,\text{K}}
%------------------------------------------------------------------------------
%       -> additional operators
%------------------------------------------------------------------------------
\newcommand{\odiv}{{\operatorname*{div}}}
\newcommand{\ograd}{{\operatorname*{grad}}}
\newcommand{\orot}{{\operatorname*{rot}}}
\newcommand{\odiag}{{\operatorname*{diag}}}
\newcommand{\ocov}{{\operatorname*{cov}}}
%------------------------------------------------------------------------------
%       -> fraction with slash instead of ratio bar
%------------------------------------------------------------------------------
\newcommand{\slashfrac}[2]{%
  \raisebox{0.5ex}{\ensuremath #1}\kern-0.12em/\kern-0.08em
  \raisebox{-.8ex}{\ensuremath #2}}
%------------------------------------------------------------------------------
%       -> shorthand for "integral limits below and above"
%------------------------------------------------------------------------------
\newcommand{\intl}{\int\limits}
\newcommand{\ointl}{\oint\limits}
%------------------------------------------------------------------------------
%       -> d'Alembert operator
%------------------------------------------------------------------------------
\newcommand{\sqr}[3]{%
    {\vcenter{\hrule height.#3ex\hbox{\vrule width.#2ex height#1ex
     \kern#1ex\vrule width.#3ex}\hrule height.#2ex}}}
\newcommand{\dalem}{\mathchoice\sqr{1.3}{08}{24}\sqr{1.4}{08}{24}
                    \sqr{1.15}{07}{21}\sqr{1.0}{06}{18}\,}
%------------------------------------------------------------------------------
%       -> vectors and matrices
%------------------------------------------------------------------------------
\newcommand{\vect}[1]{\begin{matrix}#1\end{matrix}}
\newcommand{\pvect}[1]{\begin{pmatrix}#1\end{pmatrix}}
\newcommand{\bvect}[1]{\begin{bmatrix}#1\end{bmatrix}}
\newcommand{\vvect}[1]{\begin{vmatrix}#1\end{vmatrix}}
\newcommand{\Vvect}[1]{\begin{Vmatrix}#1\end{Vmatrix}}

%% Math
\newcommand{\abs}[1]{\ensuremath{\lvert#1\rvert}} % {x}
\newcommand{\Real}{\ensuremath{\mathcal{R}e}\xspace}
\newcommand{\Imag}{\ensuremath{\mathcal{I}m}\xspace}

%% QM
\newcommand{\bra}[1]{\ensuremath{\langle #1|}}               % {a}
\newcommand{\ket}[1]{\ensuremath{|#1 \rangle}}               % {b}
\newcommand{\braket}[2]{\ensuremath{\langle #1|#2 \rangle}} % {a}{b}
\newcommand{\expect}[3]{\ensuremath{\langle #1| #2 |#3 \rangle}} % {a}{b}{c}
%------------------------------------------------------------------------------
%       -> antiparticles and particles with a bar in parentheses
%------------------------------------------------------------------------------
\newcommand{\widebar}[1]{%
   \mkern1.5mu\overline{\mkern-1.5mu#1\mkern-1.mu}\mkern1.mu}
\catcode`\@=11 % @ signs are now treated as letters
\newcommand{\parenbar}{\mathpalette\p@renb@r}
\def\p@renb@r#1#2{\vbox{%
  \ifx#1\scriptscriptstyle \dimen@.7em\dimen@ii.2em\else
  \ifx#1\scriptstyle \dimen@.8em\dimen@ii.25em\else
  \dimen@1em\dimen@ii.4em\fi\fi \offinterlineskip
  \ialign{\hfill##\hfill\cr
    \vbox{\hrule width\dimen@ii}\cr
    \noalign{\vskip-.3ex}%
    \hbox to\dimen@{$\mathchar300\hfil\mathchar301$}\cr
    \noalign{\vskip-.3ex}%
    $#1#2$\cr}}}
\catcode`\@=12 % @ signs are no longer letters
\newcommand{\nuan}{\parenbar{\nu}}
\newcommand{\nunubar}{\parenbar{\nu}}
\newcommand{\ppbar}{\parenbar{p}}
\newcommand{\qqbar}{\parenbar{q}}
\newcommand{\pbar}{\widebar{p}}
\newcommand{\nbar}{\widebar{n}}
\newcommand{\qbar}{\widebar{q}}
\newcommand{\dbar}{\widebar{d}}
\newcommand{\ubar}{\widebar{u}}
\newcommand{\sbar}{\widebar{s}}
\newcommand{\cbar}{\widebar{c}}
\newcommand{\bbar}{\widebar{b}}
\newcommand{\tbar}{\widebar{t}}
\newcommand{\nubar}{\widebar{\nu}}
\newcommand{\Dbar}{\widebar{D}}
\newcommand{\ebar}{\widebar{e}}
\newcommand{\Ebar}{\widebar{e}}
\newcommand{\Hebar}{\widebar{\text{He}}}
\newcommand{\Cbar}{\widebar{\text{C}}}
\newcommand{\Abar}{\widebar{\text{A}}}
\newcommand{\Kbar}{\widebar{K}}
\newcommand{\chibar}{\widebar{\chi}}
\newcommand{\LQbar}{\widebar{\rm LQ}}
%------------------------------------------------------------------------------
%       -> small numbers
%------------------------------------------------------------------------------
\newcommand{\zero}{{\scriptscriptstyle 0}}
\newcommand{\sone}{{\scriptscriptstyle 1}}
\newcommand{\stwo}{{\scriptscriptstyle 2}}
\newcommand{\sthr}{{\scriptscriptstyle 3}}
\newcommand{\sfou}{{\scriptscriptstyle 4}}
\newcommand{\sfiv}{{\scriptscriptstyle 5}}
\newcommand{\ssix}{{\scriptscriptstyle 6}}
\newcommand{\ssev}{{\scriptscriptstyle 7}}
\newcommand{\seig}{{\scriptscriptstyle 8}}
\newcommand{\snin}{{\scriptscriptstyle 9}}
\newcommand{\sten}{{\scriptscriptstyle 10}}
\newcommand{\half}{{\scriptscriptstyle 1/2}}
%------------------------------------------------------------------------------
%       -> abbreviations for common math mode symbols
%------------------------------------------------------------------------------
\newcommand{\MSbar}{\hbox{$\overline{\rm MS}$}\xspace}
\newcommand{\UO}{{\rm U}(1)_\sY}
\newcommand{\SUT}{{\rm SU}(2)_\sL}
\newcommand{\SUTH}{{\rm SU}(3)_c}
\newcommand{\SUTUO}{{\rm SU}(2)_\sL\times{\rm U}(1)_\sY}
\newcommand{\SUTSUTUO}{{\rm SU}(3)_c\times{\rm SU}(2)_\sL\times{\rm U}(1)_\sY}
\newcommand{\alsmu}[1]{{\alpha_s(\mu_{#1}^2)}}
\newcommand{\alsmz}{{\alpha_s(M_Z^2)}}
\newcommand{\alsqs}{{\alpha_s(Q^2)}}
\newcommand{\als}{\alpha_s}
\newcommand{\ctwb}{\cos\theta_W}
\newcommand{\ctws}{\cos^2\theta_W}
\newcommand{\diff}{{\rm d}}
\newcommand{\gh}{{\gamma_h}}
\newcommand{\lsq}[1]{{\lambda_{#1}}}
\newcommand{\lsqp}[1]{{\lambda'_{#1}}}
\newcommand{\lsqpp}[1]{{{\lambda''}_{#1}}}
\newcommand{\rnge}{\hbox{$\,\text{--}\,$}}
\newcommand{\sihat}{{\hat\sigma}}
\newcommand{\sitil}{{\tilde\sigma}}
\newcommand{\stwb}{\sin\theta_W}
\newcommand{\stwss}{\sin^4\theta_W}
\newcommand{\stws}{\sin^2\theta_W}
\newcommand{\ta}{{\theta^\ast}}
\newcommand{\tilR}{\tilde R}
\newcommand{\tilS}{\tilde S}
\newcommand{\tilU}{\tilde U}
\newcommand{\tilV}{\tilde V}
\newcommand{\tl}{{\theta_\ell}}
\newcommand{\tw}{\theta_W}
%------------------------------------------------------------------------------
%       -> roman character combinations to be used in math mode, mainly
%          sub- or superscripts
%------------------------------------------------------------------------------
\newcommand{\Born}{{\rm Born}}
\newcommand{\BR}{{\rm BR}}
\newcommand{\CC}{{\rm CC}}
\newcommand{\CI}{{\rm CI}}
\newcommand{\CJ}{{\rm CJ}}
\newcommand{\DA}{{\rm DA}}
\newcommand{\DCA}{{\rm DCA}}
\newcommand{\GRV}{{\rm GRV}}
\newcommand{\IC}{{\rm IC}}
\newcommand{\ISR}{{\rm ISR}}
\newcommand{\IP}{{\rm I$\kern-0.01667em$P}\xspace}
\newcommand{\JB}{{\rm JB}}
\newcommand{\LL}{{\rm LL}}
\newcommand{\LQ}{{\rm LQ}}
\newcommand{\LR}{{\rm LR}}
\newcommand{\MC}{{\rm MC}}
\newcommand{\NC}{{\rm NC}}
\newcommand{\NS}{{\rm NS}}
\newcommand{\QPM}{{\rm QPM}}
\newcommand{\RL}{{\rm RL}}
\newcommand{\RPv}{{{\not R}_P}}
\newcommand{\RR}{{\rm RR}}
\newcommand{\SM}{{\rm SM}}
\newcommand{\bkg}{{\rm bkg}}
\newcommand{\cut}{{\rm cut}}
\newcommand{\data}{{\rm data}}
\newcommand{\sdet}{{\rm det}}
\newcommand{\dof}{{\rm dof}}
\newcommand{\eff}{{\rm eff}}
\newcommand{\elm}{{\rm elm}}
\newcommand{\evt}{{\rm evt}}
\newcommand{\exc}{{\rm exc}}
\newcommand{\fit}{{\rm fit}}
\newcommand{\had}{{\rm had}}
\newcommand{\jet}{{\rm jet}}
\newcommand{\meas}{{\rm meas}}
\newcommand{\miss}{{\rm miss}}
\newcommand{\obs}{{\rm obs}}
\newcommand{\pnt}{{\rm pnt}}
\newcommand{\sL}{{\rm L}}
\newcommand{\sR}{{\rm R}}
\newcommand{\sS}{{\rm S}}
\newcommand{\sT}{{\rm T}}
\newcommand{\sY}{{\rm Y}}
\newcommand{\sca}{{\rm sca}}
\newcommand{\sexp}{{\rm exp}}
\newcommand{\sint}{{\rm int}}
\newcommand{\smin}{{\rm min}}
\newcommand{\smax}{{\rm max}}
\newcommand{\srad}{{\rm rad}}
\newcommand{\sthe}{{\rm the}}
\newcommand{\sys}{{\rm sys}}
\newcommand{\tot}{{\rm tot}}
\newcommand{\true}{{\rm true}}
%------------------------------------------------------------------------------
%       -> some calligraphic symbols and applications
%------------------------------------------------------------------------------
\newcommand{\Cor}{{\cal C}}
\newcommand{\F}{{\cal F}}
\newcommand{\Lumi}{{\cal L}}
\newcommand{\Prob}{{\cal P}}
\newcommand{\ord}[1]{{\cal O}(#1)}
\newcommand{\ordlr}[1]{{\cal O}\left(#1\right)}
%------------------------------------------------------------------------------
%       -> some math symbols (+,-,...) for usage as mathchar's
%------------------------------------------------------------------------------
\mathchardef\qsm=63
\mathchardef\pls=43
\mathchardef\mns=512
\mathchardef\plm=518
\mathchardef\eql=61
\mathchardef\smallleft=300
\mathchardef\smallright=301
\mathchardef\les=316
\mathchardef\gre=318
\mathchardef\leq=532
\mathchardef\grq=533
%------------------------------------------------------------------------------
%       Additional functionality for tables and figures
%       -----------------------------------------------
%       -> alignment tools for tables
%------------------------------------------------------------------------------
\newcommand{\cA}{{\phantom{0}}}
\newcommand{\cB}{{\phantom{00}}}
\newcommand{\cC}{{\phantom{000}}}
\newcommand{\cD}{{\phantom{0000}}}
\newcommand{\cAp}{{\phantom{0.}}}
\newcommand{\cM}{{\phantom{\mns}}}
\newcommand{\cP}{{\phantom{\pls}}}
%------------------------------------------------------------------------------
%       -> comments on figures (allow to add axodraw and picture items)
%------------------------------------------------------------------------------
\catcode`\@=11 % @ signs are now treated as letters
\newcounter{pict@width}
\newcounter{pict@height}
\newlength{\pict@scale}
\setlength{\pict@scale}{0.1mm}
\newcommand{\psfigadd}[4]{%
\setcounter{pict@width}{1*\ratio{#2+\pict@scale/2}{\pict@scale}}
\setcounter{pict@height}{1*\ratio{#3+\pict@scale/2}{\pict@scale}}
\setlength{\unitlength}{\pict@scale}
\hbox to #2{\hspace{-\fill}\begin{picture}(\thepict@width,\thepict@height)
\put(0,0){\psfig{figure=#1,width=#2,height=#3,clip=}}
\SetScale{0.283466457}
\SetWidth{1.763889}
{#4}
\end{picture}}
}
\newcounter{pict@widthfst}
\newcounter{pict@widthscd}
\newcounter{pict@widthtot}
\newcommand{\psfigaddtwo}[7]{%
\setcounter{pict@widthfst}{1*\ratio{#2+\pict@scale/2}{\pict@scale}}
\setcounter{pict@widthscd}{1*\ratio{#2+#4+\pict@scale/2}{\pict@scale}}
\setcounter{pict@widthtot}{1*\ratio{#2+#4+#6+\pict@scale/2}{\pict@scale}}
\setcounter{pict@height}{1*\ratio{#3+\pict@scale/2}{\pict@scale}}
\setlength{\unitlength}{\pict@scale}
\hbox{\hspace{-\fill}\begin{picture}(\thepict@widthtot,\thepict@height)
\put(0,0){\psfig{figure=#1,width=#2,height=#3,clip=}}
\put(\thepict@widthscd,0){\psfig{figure=#5,width=#6,height=#3,clip=}}
\SetScale{0.283466457}
\SetWidth{1.763889}
{#7}
\end{picture}}
}
\newcommand{\psfigror}[4]{%
\setcounter{pict@width}{1*\ratio{#2+\pict@scale/2}{\pict@scale}}
\setcounter{pict@height}{1*\ratio{#3+\pict@scale/2}{\pict@scale}}
\setlength{\unitlength}{\pict@scale}
\hbox{\begin{picture}(\thepict@width,\thepict@height)
\put(0,\thepict@height){\psfig{figure=#1,width=#3,height=#2,clip=,angle=270}}
\SetScale{0.283466457}
\SetWidth{1.763889}
{#4}
\end{picture}}
}
\newcommand{\psfigrol}[4]{%
\setcounter{pict@width}{1*\ratio{#2+\pict@scale/2}{\pict@scale}}
\setcounter{pict@height}{1*\ratio{#3+\pict@scale/2}{\pict@scale}}
\setlength{\unitlength}{\pict@scale}
\hbox{\begin{picture}(\thepict@width,\thepict@height)
\put(0,0){\psfig{figure=#1,width=#3,height=#2,clip=,angle=90}}
\SetScale{0.283466457}
\SetWidth{1.763889}
{#4}
\end{picture}}
}
\catcode`\@=12 % @ signs are no longer letters
%------------------------------------------------------------------------------
%       -> narrow figures in list environment
%------------------------------------------------------------------------------
\newlength\listtextwidth
\newcommand{\shiftfloat}[1]{
  \setlength{\listtextwidth}{\textwidth}
  \addtolength{\listtextwidth}{-#1}}
%------------------------------------------------------------------------------
%       -> "none" marker
%------------------------------------------------------------------------------
\newcommand{\none}{\hbox{---}}
%------------------------------------------------------------------------------
%       -> protected commands in captions
%------------------------------------------------------------------------------
\newcommand{\pcite}[1]{{\protect\cite{#1}}}
\newcommand{\pnl}{\protect{\newline}}
%------------------------------------------------------------------------------
%       -> pseudo-footnotes for tables
%------------------------------------------------------------------------------
\catcode`\@=11 % @ signs are now treated as letters
\newlength{\@tabfninsert}
\newlength{\@tabfnwidth}
\newcommand{\tabfootnote}[2]{%
  \setlength{\@tabfninsert}{0.8em}
  \setlength{\@tabfnwidth}{\textwidth}
  \addtolength{\@tabfnwidth}{-\@tabfninsert}
  \addtolength{\@tabfnwidth}{-0.4em}
  \noindent\makebox[\@tabfninsert][r]{\footnotesize$^{#1}$\hfil}\hfill%
  \parbox[t]{\@tabfnwidth}{\footnotesize #2\hfill}}
\catcode`\@=12 % @ signs are no longer letters


% Some other macros used in the sample text
\def\st{\scriptstyle}
\def\sst{\scriptscriptstyle}
\def\mco{\multicolumn}
\def\epp{\epsilon^{\prime}}
\def\vep{\varepsilon}
\def\ra{\rightarrow}
\def\ppg{\pi^+\pi^-\gamma}
\def\vp{{\bf p}}
\def\ko{K^0}
\def\kb{\bar{K^0}}
\def\al{\alpha}
\def\ab{\bar{\alpha}}
\def\be{\begin{equation}}
\def\ee{\end{equation}}
\def\bea{\begin{eqnarray}}
\def\eea{\end{eqnarray}}
\def\CPbar{\hbox{{\rm CP}\hskip-1.80em{/}}}



\def\BsDsH{B_{s}\to D_{s} h}

\def\BsDsPi{B_{s}\to D_{s} \pi}
\def\BsDsK{B_{s}\to D_{s} K}

\def\Bpi{B_{s}\to D_{s}\pi}
\def\Bpiminus{B_{s}^{0}\to D_{s}^{-}\pi^{+}}
\def\Bbarpiminus{\bar{B}_{s}^{0}\to D_{s}^{-}\pi^{+}}
\def\Bbarpiplus{\bar{B_{s}^{0}}\to D_{s}^{+}\pi^{-}}
\def\DsPiminus{D_{s}^{-}\pi^{+}}
\def\DsPiplus{D_{s}^{+}\pi^{-}}

\def\BK{B_{s}\to D_{s}K}
\def\BKminusplus{B_{s}^{0}\to D_{s}^{\mp}K^{\pm}}
\def\BbarKplusminus{\bar{B}_{s}^{0}\to D_{s}^{\pm}K^{\mp}}
\def\BKminus{B_{s}^{0}\to D_{s}^{-}K^{+}}
\def\BKplus{B_{s}^{0}\to D_{s}^{+}K^{-}}
\def\BbarKminus{\bar{B_{s}^{0}}\to D_{s}^{-}K^{+}}
\def\BbarKplus{\bar{B_{s}^{0}}\to D_{s}^{+}K^{-}}

\def\dMs{\Delta m_{s}}
\def\resdMs{\sigma(\Delta m_{s})}
\def\weak{\gamma +\phi_s}
\def\resweak{\sigma(\gamma +\phi_s)}
\def\strong{\Delta_{T1/T2}}
\def\resstrong{\sigma(\Delta_{T1/T2})}
\def\pt{\tau_{rec}}
\def\pterror{\Delta\tau_{rec}}
\def\dgammaovergamma{\Delta \Gamma_{s}/\Gamma_{s}}

\def\normlamsq{|\lambda_{f}|^{2}}
\def\normlambarsq{|\bar{\lambda}_{\bar{f}}|^{2}}
\def\deltag{\frac{\Delta \Gamma_{s} \, t}{2}}
\def\dMst{\Delta m_{s} \, t}
\def\poverq{\Big|\frac{p}{q}\Big|^{2}}
\def\qoverp{\Big|\frac{q}{p}\Big|^{2}}

\def\ADG{A^{\Delta\Gamma}_{f}}
\def\ADGb{A^{\Delta\Gamma}_{\bar{f}}}
